\section{Simulator}

\begin{frame}{Sensor Data Exchange 1}
\begin{itemize}

\item Data exchange in real-time using TCP
\vspace{12pt} 

\item \textbf{Problem:} Nagle's algorithm

It aims to gain efficiency by reducing the number of packets that need to be sent over the network. 

\item \textbf{Solution:} 
    \begin{itemize}
    \item Set socket option \texttt{TCP\_NODELAY} to disable Nagle's algorithm
    \end{itemize}
\vspace{12pt} 
    
\item \textbf{Problem:} Small sliding window

Window size is the amount of unacknowledged data that can be in transit at any given time.
\item \textbf{Solution:} 
    \begin{itemize}
    \item Buffer the sensor data before sending
    \end{itemize}
\end{itemize}
\end{frame}

\begin{frame}{Sensor Data Exchange 2}
\begin{itemize}

\item \textbf{Problem:} Sensor data syncronisation

The application doesn't know which radar detection was made at which location.

\item \textbf{Solution:} 
    \begin{itemize}
    \item Send GNSS and Radar detections combined as pairs
    \end{itemize}

\vspace{12pt} 
\item \textbf{Problem:} Split data pairs

TCP is a byte stream protocol that doesn't preserve boundaries. \texttt{OS\_Socket\_read} could receive only half of a pair.

\item \textbf{Solution:} 
    \begin{itemize}
    \item Add size before every pair
    \end{itemize}
 

\end{itemize}
\end{frame}

\begin{frame}{Park Spot Detection \& Parking}
\begin{itemize}


\item \textbf{Assumptions:} 
    \begin{itemize}
    \item Same street, Same vehicles, Parking space between two vehicles
    \end{itemize}
\vspace{12pt} 
\item Compare each detection with last detection to get distance difference
\item If distance is big enough, parking space is found
\vspace{12pt} 

\item \textbf{Parking:} It is a fixed procedure and consists of 4 steps
    \begin{itemize}
    \item Stop Vehicle if parking space is found.
    \item Do a right lane change
    \item Drive backwards to the vehicle above parking space
    \item Park vehicle

    

    \end{itemize}    

\end{itemize}
\end{frame}