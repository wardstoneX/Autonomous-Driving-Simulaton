\section{TPM Driver}

\begin{frame}{Concept}
\begin{itemize}
\item WolfTPM handles the specifics of communicating with the TPM
\item But it doesn't know how to do input/output on TRENTOS
  \begin{itemize}
  \item[$\Rightarrow$] Need to add custom I/O function to the HAL (``Hardware Abstraction Layer'')
  \item Custom I/O wrapper actually just a thin wrapper around the BCM2837 library
  \end{itemize}
\item It also doesn't have a CAmkES interface that would let it talk to other TRENTOS components
  \begin{itemize}
  \item[$\Rightarrow$] That's what our custom interfaces are for
  \end{itemize}
\end{itemize}
\end{frame}

\begin{frame}{Custom Interfaces}
\begin{itemize}
\item Larger data, e.g. keys or encrypted text, get passed over the dataport
\item Functions only take the parameters needed to process the data
\item[$+$] No additional buffer needed inside \texttt{trentos.c}\\
    Dataport pointer can immediately be passed to WolfTPM
\item[$+$] Avoids unnecessary copying of data
\item[$-$] Dataport doesn't show up in function signature
\item[$-$] Programmer of the client must remember to copy data out when it would otherwise be overwritten
\end{itemize}
\end{frame}

\begin{frame}[fragile]{Custom Interfaces}
\texttt{if\_KeyStore}: Functions for getting or storing keys

%/* Functions to get cEK and cSRK */
%/* Functions to store or get any other key */
\begin{minted}{c}
void     getCEK_RSA2048(uint32_t *exp);
void     getCSRK_RSA1024(uint32_t *exp);
uint32_t storeKey(uint32_t keyLen);
int      loadKey(uint32_t hdl, uint32_t *keyLen);
\end{minted}

Note:
\begin{itemize}
\item Nothing requires the data that's stored to actually be a key
\item ``Handle'' actually just an offset, no validity checks
\end{itemize}
\end{frame}

\begin{frame}[fragile]{Custom Interfaces}
\texttt{if\_Crypto}: Functions for encryption and decryption

\begin{minted}{c}
int decrypt_RSA_OAEP(int key, int *len);
int encrypt_RSA_OAEP(int key, int *len);
\end{minted}

Note that this signature allows to call multiple decryption operations after each other!

\begin{minted}{c}
crypto.decrypt_RSA_OAEP(IF_CRYPTO_KEY_CEK, &len);
crypto.decrypt_RSA_OAEP(IF_CRYPTO_KEY_CSRK, &len);
\end{minted}
\end{frame}

\begin{frame}[fragile]{Custom Interfaces}
\texttt{if\_TPMctrl}: Functions to control the TPM itself

\begin{minted}{c}
void shutdown(void)
\end{minted}

Why is this needed?

\begin{itemize}
\item TPM goes into Dictionary Attack Lockdown mode if improperly shut down too often
\item CAmkES appears not to have a \texttt{pre\_shutdown()} equivalent of \texttt{pre\_init()}
\item \texttt{SecureCommunication} also a passive component\\
\item[$\Rightarrow$] Shutdown command must be sent by \texttt{TestApp}, can't re-use interfaces already connected to \texttt{SecureCommunication}
\end{itemize}
\end{frame}

\begin{frame}{Initialization}
\begin{itemize}
\item Upon component startup, \texttt{pre\_init()} gets called
\item Also possible to define component-specific init functions and \texttt{post\_init()}, but not needed for this implementation
\end{itemize}

Our \texttt{pre\_init()} function:

\begin{enumerate}
\item Initializes BCM2837 SPI with correct parameters (MSB first, CS1, 31.25 MHz)
\item Pulls $\overline{\mbox{RESET}}$ pin low
\item Generates real SRK (needed for generation of other keys and for NV storage)
\item Loads cEK into TPM\ldots
\item Generates cSRK
\end{enumerate}
\end{frame}

\begin{frame}{Loading cEK}
Problem:
\begin{itemize}
\item First run of the code must save cEK into TPM
\item Python client also must receive the cEK, for verification
\item Subsequent runs must load this cEK, not generate a new one!
\end{itemize}
\end{frame}

\begin{frame}{Loading cEK}
How to determine whether cEK already exists?
\begin{itemize}
\item \texttt{pre\_init()} tries to open NV storage
\end{itemize}

If it succeeds:
\begin{itemize}
\item Read the cEK from NV storage into a \texttt{WOLFTPM2\_KEYBLOB} struct
\item Load the cEK into TPM main memory
\end{itemize}
\end{frame}

\begin{frame}{Loading cEK}
If it fails:

\begin{itemize}
\item Generate new cEK (as a \texttt{WOLFTPM2\_KEYBLOB} struct)
\item Store the struct it in the NVRAM of the TPM
\item Print a hexdump of the new key and exit
\item Python client can import cEK using \texttt{importEK.py}
\end{itemize}

How to clear TPM and re-create cEK on purpose?

\begin{itemize}
\item Define \texttt{CLEAR\_TPM} and re-compile
\item That will clear the TPM and re-create the key
\end{itemize}
\end{frame}

\begin{frame}{Loading cEK}
Additional difficulty: a \texttt{WOLFTPM2\_KEYBLOB} is too large!

\begin{itemize}
\item One NV index can only store 2048 bytes
\item Somehow, \texttt{sizeof(WOLFTPM\_KEYBLOB)} is \ttilde{}2300 bytes!
\item[$\Rightarrow$] Our solution: use two NV indices
\item[$\Rightarrow$] Better solution would've been to find out why that struct is wasting so much space
\item[$\Rightarrow$] Time constraints made us choose worse solution\ldots
\end{itemize}
\end{frame}
